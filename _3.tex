\begin{MyExercise}
La categoria $\Delta$ che ha per oggetti gli insiemi totalmente ordinati e finiti non \`e cartesiana (come dovrebbe essere fatto il prodotto?)
\end{MyExercise}
La categoria $\bf \Delta$ non è cartesiana. Infatti, dalla monotonicità delle proiezioni
\[(a,b)\leq(a',b') \Rightarrow a\leq a' \wedge b \leq b'\]
e dalla totalità dell'ordine del prodotto
\[(a,b)\leq(a',b')\vee(a',b')\leq(a,b)\]
seguirebbe
\[
a\leq a' \wedge b \leq b'
\bigvee
a'\leq a \wedge b' \leq b
\]
ovvero
\[
a\leq a' \vee b' \leq b
\bigwedge
a'\leq a \vee b \leq b'
\]
che non è sempre vero.
\fou[inline]{Dunque; ti dico come l'ho fatto io. Dentro $\boldsymbol \Delta$ stanno gli insiemi \emph{totalmente} ordinati con funzioni monotone annesse. Il prodotto cartesiano \`e totalmente ordinato dall'ordine lessicografico, quello sulle componenti non ordina totalmente $[1]\times [1]$ ($(1,0)$ e $(0,1)$ non sono confrontabili!), e per\`o rispetto all'ordine lessicografico le proiezioni canoniche non sono monotone --avrebbero bisogno di quello sulle componenti-- Insomma, \`e il lemma della coperta troppo corta.}
\pol[inline]{Sì, stiamo dicendo esattamente la stessa cosa. Infatti $(1,0)$ e $(0,1)$ sono proprio due elementi che rendono falsa la condizione che ho scritto - ovvero, sui quali supporre che valgano congiuntamente totalità e monotonia delle proiezioni porta ad un assurdo. Infatti la cosa che ho scritto vuol dire che non è possibile ordinare elementi del prodotto le cui componenti non siano ordinate concordemente tra loro.}
