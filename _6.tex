\begin{MyExercise}
Sia $\C$ una qualsiasi categoria piccola; mostrare che esiste una aggiunzione $\Delta\dashv \Gamma$, dove $\Delta\colon \Sets \to \Sets^{\C^\text{op}}$ \`e il funtore costante in $X\in\Sets$, $\Delta_X(C) \equiv X$ per ogni $C\in\C$ (hint: $\Delta = t^*\colon \Sets^{\{\bullet\}} \to \Sets^{\C^\text{op}}$ se $t\colon\C\to\{\bullet\}$ \`e il funtore terminale); descrivere l'azione di questo $\Gamma$ su oggetti e morfismi di $\Sets^{\C^\text{op}}$.
\end{MyExercise}

Un funtore $F\in[\C,\D]$ induce un funtore tra le categorie dei prefasci $F^\star\in[\textsf{PSh}(\D),\textsf{PSh}(\C)]$ dato dalla precomposizione con il funtore speculare $\tilde{F}\in[\C^{\bf op},\D^{\bf op}]$ definito come $\tilde{F}:=(-)^{\bf op} \circ F \circ (-)^{\bf op}$.
\fou[inline]{S\`i, quello che stai dicendo \`e che esiste un(o pseudo)funtore $(-)^\star\colon \mathbf{Cat}\to \mathbf{Cat}$ che manda una categoria -fottiamocene delle taglie, o meglio, fissa due universi incapsulati- nella sua cat dei prefasci; come agisce sui morfismi di $\bf Cat$ tale pseudofuntore? Esattamente mandando $F$ nel tuo $F^\star$. Ora, in quanto pseudofuntore esso rispetta le coppie aggiunte: questo \`e quello che ti impegni a mostrare dopo, ovvero che se $F\dashv G$ in $\bf Cat$, allora $G^*\dashv F^*$. Tu ti metti a farlo sui soli rappresentabili, ma in realt\`a \`e vero pi\`u in generale in un modo pi\`u astratto: semplicemente dati $F\colon \bf C\to D$, con un aggiunto destro $G\colon \bf D\to C$ si ha che $\textsf{Lan}_F\cong G^\star$ (ricorda che $\textsf{Lan}_F\dashv F^\star$ \emph{per definizione} di $\textsf{Lan}_F$!):
\begin{align*}
\textsf{Lan}_F(H) &\cong \int^A \hom_\D(FA, -)\times HA \\
(agg.) & \cong \int^A \hom_\D(A, G(-))\times HA\\
(Yon)&\cong HG(-)
\end{align*}
dove il primo step usa $F\dashv G$ e il secondo Yoneda ninja.}

Se $G\in[\D,\C]$, usando Yoneda e denotando con $\hat{X}$ ($\check{X}$) il (co)prefascio (co)rappresentato da $X$ otteniamo due cascate di oggetti isomorfi:
\[
\begin{array}{c}
\D^{\bf op}( D, \tilde{F}C ) \\
\D( FC, D ) \\
\textsf{PSh}(\C)( \C(-,C), \D(-,D) \circ \tilde{F} ) \\
\textsf{PSh}(\C)( \hat{C}, F^\star\hat{D} ) \\
\textsf{CoPSh}(\C)( \tilde{F}^\star\check{D}, \check{C} )
\end{array}
\qquad
\begin{array}{c}
\C^{\bf op}( \tilde{G}D, C ) \\
\C( C, GD ) \\
\textsf{CoPSh}(\D)( \D^{\bf op}(D,-), \C^{\bf op}(C,-) \circ G ) \\
\textsf{CoPSh}(\D)( \check{C}, \tilde{G}^\star\check{D} ) \\
\textsf{PSh}(\D)( G^\star\hat{C}, \hat{D} )
\end{array}
\]
dove gli isomorfismi inferiori sono conseguenza del fatto che la corappresentabilità equivale alla rappresentabilità nella categoria opposta.

Unendo le colonne otteniamo quattro condizioni di aggiuntezza equivalenti:
\[
\begin{array}{c}
\tilde{G} \dashv \tilde{F} \\
F \dashv G \\
G^\star \dashv F^\star \text{ sui rappresentabili}\\
\tilde{F}^\star \dashv \tilde{G}^\star \text{ sui corappresentabili}\\
\end{array}
\]

Bene. Ora facciamo l'esercizio.

Sia $\bf C$ una categoria piccola e $t:\C\rightarrow\{\bullet\}$ l'unico funtore da essa verso l'Occhio di Sauron, la categoria terminale. I prefasci sull'Occhio sono funtori che selezionano ognuno dei set e la relativa identità. Precomporne uno con $\tilde{t}$ restituisce un prefascio che mappa l'intera $\C^{\bf op}$ nell'oggetto e nell'identità che il primo individuava. Dunque essenzialmente $t^\star=\Delta$. Ma allora se esiste un'aggiunzione $\Delta\dashv\Gamma$ esiste anche $s\dashv t$ ed almeno sui rappresentabili $\Gamma=s^\star$. 
\fou[inline]{Come ti dicevo, il punto \`e:
\begin{itemize}
\item Non sempre $\bf C$ ha un iniziale: in tal caso $\Gamma$ esiste ma $s$ no; dunque
\item Non ogni aggiunzione tra categorie di prefasci \`e indotta da un'aggiunzione nel senso superiore: questo in gergo si dice ``il 2-funtore $(-)^\star$ non \`e locally full'', perch\'e ogni funtore $F^\star$ ha $\textsf{Lan}_F$, e d'altra parte non \`e sempre vero che ogni $\textsf{Lan}_F$ \`e isomorfa a $G^*$ per qualche $G$ (controesempio? questo va benissimo: $\bf C$ sarebbe forzata ad avere un iniziale).
\end{itemize}}
Chi è, dunque, $s$? L'aggiunzione si riduce a
\[
\C(s\bullet,B)
\cong
\{\bullet\}(\bullet,tB)
\cong
\{\bullet\}(\bullet,\bullet)
\cong
\star
\]
ma ciò significa che $s\bullet$ dev'essere l'oggetto iniziale di \C, la cui esistenza è garantita dal fatto che $s$ sia aggiunto sinistro e $\bullet$ sia iniziale. Per funtorialità si completa la definizione di $s$: esso individua l'oggetto iniziale di \C e la sua identità. Come agisce $s^\star$? Precomporre $\tilde{s}$ ad un prefascio su \C dà un prefascio che individua l'immagine dell'oggetto iniziale tramite il primo. Questo dà un'aggiunzione, ma solo sui prefasci rappresentabili! Infatti è immediato che in generale
\[\begin{array}{c}
\textsf{PSh}(\C)( \Delta_P, Q )
\cong
\textsf{PSh}(\C)( t^\star P, Q )
\ncong
\textsf{PSh}(\{\bullet\})( P, s^\star Q )
\cong
\Sets( P, Qs\bullet )
\end{array}\]

\pol[inline]{Il conto pertanto non ha risolto l'esercizio ma era istruttivo ed ha fornito un buon controesempio, come dicevamo sopra. Ora ricominciamo daccapo.}


Abbiamo bisogno della seguente
{\prop L'estensione di Kan destra (sinistra) lungo il funtore verso la categoria terminale è l'operazione di (co)limite.}
{\proof
Srotoliamo le definizioni. Detto $t\colon\C\to\{\bullet\}$ definiamo la precomposizione $t^\star\colon[\{\bullet\},\D]\to[\C,\D]$. Le estensioni di Kan allora ammontano alle aggiunzioni
\[\begin{array}{rcccl}
&&		[\C,\D] (t^\star D,F)
&\cong&	[\{\bullet\},\D] (D,{\textsf Ran}_tF)
\\\null	[\{\bullet\},\D] ({\textsf Lan}_tF,D)
&\cong&	[\C,\D] (F,t^\star D)
\end{array}\]
Se $D:\bullet\mapsto d$ allora $t^\star D=\Delta_d$, il funtore costante su $\C$ con immagine $d$. Quindi
\[\begin{array}{rcccl}
&&		[\C,\D] (\Delta_d,F)
&\cong&	\D (d,{\textsf Ran}_tF\bullet)
\\\null	\D ({\textsf Lan}_tF\bullet,d)
&\cong&	[\C,\D] (F,\Delta_d)
\end{array}\]
ovvero l'estensione sinistra (destra) (co)rappresenta i (co)coconi.
Dunque per definizione di (co)limite
\[
{\textsf Lan}_t(-)\bullet \cong {\rm colim}
\qquad
{\textsf Ran}_t(-)\bullet \cong {\rm lim}
\]
\qed}

Per quanto detto, $\Gamma={\textsf Ran}_t(-)\bullet$ ed è ben definito poiché $\C^\text{op}$ è piccola e $\Sets$ è completa. $\Gamma$ manda funtori nel loro limite e trasformazioni naturali nella freccia tra i limiti di dominio e codominio univocamente determinata per naturalità.
\fou[inline]{Bang! Molto bene. Adesso come esercizio bonus, usa il calcolo delle cofini per mostrare lo stesso risultato:
\[
\text{Lan}_tK\cong \int^C \hom_{\{\bullet\}}(tC,-)\times KC\qquad 
\text{Ran}_tK\cong \int_C KC^{\hom(-, tC)}
\]}

\pol[inline]{Ok.}

{\prop Se tutte le categorie coinvolte sono piccole ed il codominio dei funtori da estendere è (co)completo \pol{Basta meno, ma non mi sbilancio.} allora le estensioni di Kan esistono.}
{\proof
\[\begin{array}{rl}
&[\cate{B},\D](F,L^\star G)
\\\cong&
[\cate{B},\D](F,GL)
\\\cong&
\int_{b:\cate{B}} \D(Fb,GLb)
\\\cong&
\int_{b:\cate{B}} [\D,\Sets](\D(Lb,-),\D(Fb,G-))
\\\cong&
\int_{b:\cate{B}}\int_{s:\Sets} \Sets(\D(Lb,s),\D(Fb,Gs))
\\\cong&
\int_{b:\cate{B}}\int_{s:\Sets} \D(\D(Lb,s)\times Fb,Gs)
\\\cong&
\int_{s:\Sets}\int_{b:\cate{B}} \D(\D(Lb,s)\times Fb,Gs)
\\\cong&
\int_{s:\Sets} \D(\int^{b:\cate{B}}\D(Lb,s)\times Fb,Gs)
\\\cong&
[{\mathbf E},\D](\int^{b:\cate{B}}\D(Lb,-)\times Fb,G)
\end{array}
\begin{array}{rl}
[\C,\D](R^\star F,G)
&\cong\\\null
[\C,\D](FR,G)
&\cong\\
\int_{c:\C} \D(FRc,Gc)
&\cong\\
\int_{c:\C} [\D,\Sets](\D(-,Rc),\D(F-,Gc))
&\cong\\
\int_{c:\C}\int_{s:\Sets} \Sets(\D(s,Rc),\D(Fs,Gc))
&\cong\\
\int_{c:\C}\int_{s:\Sets} \D(Fs,Gc^{\D(s,Rc)})
&\cong\\
\int_{s:\Sets}\int_{c:\C} \D(Fs,Gc^{\D(s,Rc)})
&\cong\\
\int_{s:\Sets} \D(Fs,\int_{C:\C}Gc^{\D(s,Rc)})
&\cong\\\null
[{\mathbf F},\D](F,\int_{C:\C}Gc^{\D(-,Rc)})
\end{array}\]
e dunque
\[
\text{Lan}_LF\cong \int^{b:\cate{B}}\D(Lb,-)\times Fb
\qquad 
\text{Ran}_RG\cong \int_{C:\C}Gc^{\D(-,Rc)}
\]
\qed}

{\oss Se $L\dashv R'$ ed $L'\dashv R$ allora
\[\begin{array}{rcccl}
\int^{b:\cate{B}}\D(Lb,-)\times Fb
&\cong&
\int^{b:\cate{B}}\D(b,R'-)\times Fb
&\cong&
FR'\\
\int_{C:\C}Gc^{\D(-,Rc)}
&\cong&
\int_{C:\C}Gc^{\D(L'-,c)}
&\cong&
GL'
\end{array}\]
e pertanto
\[L\dashv R \qquad\Leftrightarrow\qquad R^\star\dashv L^\star\]
Ciò però si può mostrare anche chiedendo soltanto che le categorie siano tutte piccole.
}