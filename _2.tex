\begin{MyExercise}
La categoria dei poset (morfismi: funzioni monotone) è cartesiana chiusa; la categoria degli spazi metrici limitati (quelli dove $\sup_{x,y\in X}d(x,y)<\infty$, e con morfismi le funzioni lipschitziane) è cartesiana chiusa; 
\end{MyExercise}

La categoria $\cate{Pos}$ ha come oggetti i poset, ovvero set dotati di relazioni binarie riflessive, antisimmetriche e transitive. Esse sono preservati dalle funzioni monotone, i morfismi della categoria.

\pol[inline]{Domanda: $\Sets$ è cartesiana chiusa. Ciò permette di inferire qualcosa in generale su di una categoria concreta se esiste un'aggiunzione free-forgetful?}

Come sono fatti i prodotti, se ci sono?

\pol[inline]{$\Sets$ è una sottocategoria piena di $\cate{Pos}$: i poset con relazioni vuote. Questo mi fa sospettare che ogni prodotto in $\cate{Pos}$ sia un prodotto di $\Sets$ dotato di una relazione adeguata a garantire monotonicità al $\Sets$-prodotto cartesiano di morfismi monotoni. So che funziona, ma non so dire se sia un caso.}
\hypertarget{limit-creation}{}
\fou[inline]{Non \`e un caso: esiste la nozione di \emph{creare i limiti} per un funtore $U\colon \C\to Set$, la definizione precisa \`e su \textbf{CWM.V.1}. Siccome la definizione di creazione di limiti \`e sempre piuttosto dolorosa, provo a spiegarla in altri termini.  Idealmente significa che esiste un modo canonico di dare a un universale in $\Sets$ una $\C$-struttura che lo renda un universale in $\C$; in altre parole si tratta di un modo formale di dire che ``il prodotto di gruppi \`e il prodotto cartesiano dotato delle operazioni pointwise'', oppure che ``lo spazio topologico prodotto \`e il prodotto cartesiano con la massima topologia che rende continue le proiezioni insiemistiche''. Solitamente funtori monadici creano i limiti (Prop. 1 p. 178 di \emph{Sheaves in Geometry and Logic}, ma si dimostra anche a mano). Allora siccome i forgetful $U\colon \C\to Set$ da categorie algebriche sono tutti monadici (\textbf{CWM.VI.8}, Thm. 1)\dots

Sono contento tu l'abbia notato. Non \`e una cosa banale (per apprezzarla ci ho messo quasi tutta la tesi triennale).}

Dati due poset $A,B$ definiamo il loro prodotto come il prodotto cartesiano dei set soggiacenti, dotato della relazione
\[(a,b)\leq(a',b')\Leftrightarrow(a\leq a'\wedge b\leq b')\]

\pol[inline]{Questo prodotto quindi eredita dalla congiunzione logica l'associatività. Ci piace.}

Con l'oggetto terminale di $\Sets$, il singoletto $1$, si costruisce facilmente quello di $\cate{Pos}$, ovvero $1$ dotato della relazione $1^2\cong 1$ \sout{(l'unica altra possibile è $\emptyset$ ma impedisce all'oggetto di essere terminale: non esistono morfismi da una relazione non vuota a quella vuota)}.\fou[inline]{Soprattutto perch\'e la relazione vuota non \`e un ordine parziale!\pol[inline]{Giusto. La riflessività fa in modo che il vuoto sia l'unico set ad ammettere il vuoto come relazione.}} È immediato che si tratta dell'identità del prodotto, poiché trivializza la condizione logica corrispondente alla sua componente nella disgiunzione della definizione di prodotto.

\pol[inline]{A margine noto che l'oggetto iniziale è $\emptyset$ dotato della relazione $\emptyset^2\cong\emptyset$. Non esiste oggetto zero poiché non esistono morfismi dall'oggetto terminale all'oggetto terminale.}

Il corretto prodotto di morfismi è esattamente quello di $\Sets$. Infatti l'ordine definito sul prodotto garantisce che il prodotto di funzioni monotone sia monotono, essendo
\[(f\times g)(a,b)\leq(f\times g)(a',b')\Leftrightarrow(f(a)\leq f(a')\wedge g(b)\leq g(b'))\]

Notiamo ora che la monotonia di un morfismo $f:A\times B\rightarrow C$ implica $\forall b\in B$ che
\[a\leq a' \Rightarrow f(a,b)\leq f(a',b)\]
e quindi l'isomorfismo di currying induce grazie all'ordine sui prodotti l'ordine puntuale sugli hom-set. Precisamente, se $f,g\in B^A$
\[f\leq g \Leftrightarrow f(a)\leq g(a) \quad \forall a\]
Ciò significa che abbiamo anche gli esponenziali e la categoria è cartesiana chiusa.

Cristallizziamo le idee. Fissato un oggetto $B$ di $\cate{Pos}$ abbiamo l'aggiunzione $(-\times B)\dashv(-^B)$, i cui funtori danno prodotti ed esponenziali di $\Sets$ dotati delle relazioni viste sopra. Mostriamolo presentando esplicitamente le bijezioni naturali tra gli hom-set.

Per ogni coppia di oggetti $A$ e $C$ definiamo
\[
\xymatrix{ {\rm hom}(A\times B,C)
\ar[rr]^{\gamma_{A,C}} &&
{\rm hom}(A,C^B)}
\]
e la sua inversa come
\[
(\gamma_{A,C}f)(a)(b):=f(a,b)
\qquad
(\gamma_{A,C}^{-1}g)(a,b):=g(a)(b)
\]
per ogni $(a,b)\in A\times B$. Si tratta del currying di $\Sets$ ed ogni verifica sulla bijettività è immediata. Rimane da vedere se è naturale. Dati tre morfismi $f:Z\rightarrow A$, $g:C\rightarrow X$ ed $h\in{\rm hom}(A\times B,C)$, per ogni $z\in Z$ e $b\in B$ si ha
\[
\left[ {\rm hom}(f,g^B) \circ \gamma_{A,C} (h) \right] (z)(b) = 
g\left(\gamma_{A,C}h(f(z))(b)\right) = 
g\circ h(f(z),b)
\]
\[
\left[ \gamma_{Z,X}\circ{\rm hom}(f\times B,g) (h) \right] (z)(b) =
\gamma_{Z,X} \left[ g\circ h\circ(f\times{\rm id}_B) \right] (z)(b)=
g\circ h(f(z),b)
\]
e così concludiamo.
\pol[inline]{Ew. Vabè.}

\subsection*{Spazi metrici limitati e funzioni lipschitziane}

La categoria in esame ha come oggetti gli spazi metrici limitati, ovvero le coppie $(X,d)$ dove $X$ sia un set e $d$ una metrica su di esso che abbia la proprietà
\[\sup_{x,x'\in X}d(x,x')<\infty\]
I morfismi sono le funzoni lipshitziane. Cioè, un morfismo $f:(X,d_X)\rightarrow(Y,d_Y)$ è una funzione tra i set soggiacenti per cui esiste una costante non negativa $K_f$ tale che per ogni coppia di punti $(x,x')$ del dominio vale la seguente condizione metrica:
\[d_Y(f(x),f(x'))\leq K_fd_X(x,x')\]
Le verifiche degli assiomi di categoria sono straightforward.

\pol[inline]{Un fatto grazioso è che si possa scegliere $K_{f\circ g}=K_fK_g$. Cioè, l'assegnazione della costante di Lipschitz minima è funtoriale!}
\hypertarget{lipschitz-functor}{}
\fou[inline]{Interessante, non ci avevo mai pensato; che interpretazione ha la costante di Lipschitz in termini di spazi metrici come categorie arricchite?}
Procediamo a verificare se è cartesiana chiusa.

\pol[inline]{Nuovamente, non faccio altro che pigliare quello che funzionava in $\Sets$ e dotarlo della giusta struttura aggiuntiva. Possibile che sia un caso? Di nuovo, $\Sets$ è una sottogategoria piena, data dagli spazi metrici con metriche costanti nulle, tra cui tutte le funzioni sono lipschitziane.}

L'oggetto terminale è il set di cardinalità uno, dotato dell'unica metrica possibile.

Ingenuamente, decido che il prodotto tra spazi metrici sia il prodotto cartesiano dei set soggiacenti dotato della giusta $p$-metrica. L'oggetto terminale è l'unità indipendentemente dalla scelta di $1\leq p\leq \infty$.

\pol[inline]
{Capisco che $p=\infty$ sia la scelta più semplice, ma non ho un motivo concreto perpreferirla. }
\fou[inline]
{Attento che ci sono spazi metrici la cui metrica non \`e una $p$-norma. \`E pi\`u semplice di quel che pensi mettere una metrica su $X\times Y$. E c'\`e anche un modo di metterla su $\prod_{i=1}^\infty X_i$\dots}
\pol[inline]
{Certo, ma il modo in cui definisco la metrica sul prodotto mica vincola l'esistenza di oggetti con metriche differenti. O no? Comunque, ho intuito come deve andare a finire. È che per questo esercizio invece che costruire prodotti ed esponenziali e mostrare l'aggiunzione voglio provare a capire come gli esponenziali inducono i prodotti, supponendo che l'aggiunzione esista. For fun. Ed anche perché presumo ci saranno situazioni in cui è più semplice definire la struttura pointwise sugli esponenziali ed indurre quella sui prodotti, invece che presentarle entrambe e sperare che combacino.}
\fou[inline]
{Allora ricapitolando: dati $(X,d_X)$ e $(Y, d_Y)$ definiamo
\[
d_{X\times Y}((x,y), (x', y')) = N(d_X(x,x'), d_Y(y,y'))
\]
dove $N\colon \mathbf{R}\times\mathbf{R}\to \mathbf{R}$ \`e una qualsiasi metrica che induca la topologia euclidea su $\mathbf R$. Il problema di stabilire quale di queste scelte sia canonica probabilmente \`e mal posto, perch\'e ci sar\`a in ogni caso un'unica isometria tra due scelte $N$ ed $N'$.}
\pol[inline]
{Ok. But let me take a detour.}




\subsection*{Pointwise to componentwise - and back!}

Immaginiamo di avere una categoria concreta e cartesiana chiusa i cui morfismi debbano rispettare una certa proprietà $\mathcal G$, nel senso che
\[f\in[A,B] \quad\Leftrightarrow\quad {\mathcal G}(f)\]

L'isomorfismo naturale dell'aggiunzione altri non è che il currying:

\[
\left[ A\times B, C \right]
\ni f \mapsto \hat{f} \in
\left[ A, C^B \right]
\]

Immaginiamo ora che sugli esponenziali la proprietà morfismo sia definita puntualmente, ovvero

\[
{\mathcal G}(\hat{f})
\quad\Leftrightarrow\quad
(\forall b\in B){\mathcal G}(\hat{f}(-)(b))
\]

Segue che
\begin{align*}
\hat{f} \in \left[ A, C^B \right]
& \quad\Leftrightarrow\quad
{\mathcal G}(\hat{f}) \bigwedge (\forall a\in A){\mathcal G}(\hat{f}(a)) \\
& \quad\Leftrightarrow\quad
(\forall b\in B){\mathcal G}(\hat{f}(-)(b)) \bigwedge (\forall a\in A){\mathcal G}(\hat{f}(a)(-))
\end{align*}
ovvero
\begin{align*}
f \in \left[ A \times B, C \right]
& \quad\Leftrightarrow\quad
(\forall b\in B){\mathcal G}(f(-,b)) \bigwedge (\forall a\in A){\mathcal G}(f(a,-))
\end{align*}

{\bf Morale:} {\it pointwise} sugli esponenziali corrisponde naturalmente a {\it componentwise} sui prodotti. Ciò permette di indurre anche le strutture degli oggetti da un lato all'altro dell'aggiunzione - precisamente, usando l'ultimo {\it ovvero}.
\hypertarget{componentwise-pointwise}{}
\fou[inline]{Tutto ci\`o \`e molto bello e non ci avevo mai pensato; non so se la propriet`a abbia un nome; non credo non sia nota. Posti dove cercherei:
\begin{itemize}
\item Lambek, Joachim, and Philip J. Scott, eds. \emph{Introduction to higher-order categorical logic}. Vol. 7. Cambridge University Press, 1988.
\item Barr, Michael, and Charles Wells. \emph{Toposes, triples and theories}. Vol. 278. New York: Springer-Verlag, 1985.
\end{itemize}
Forse c'entra l'aggiunzione $\exists\dashv \forall$? Ovviamente la parte successiva la seguo, ma l'orgetta di quantificatori rende tutto abbastanza illeggibile.}




\subsection*{Ordini parziali e funzioni monotone - REDUX}

Facciamo un esperimento con $\cate{Pos}$. La condizione di morfismo è la monotonicità, e per parlarne bastano le strutture di poset su domino e codominio (ovviamente, visto che si tratta esclusivamente di rispettarle):
\[
f \in [A,B]
\quad\Leftrightarrow\quad
{\mathcal G}(f)
\quad\Leftrightarrow\quad
(\forall a,a' \in A) (a \leq a' \Rightarrow f(a) \leq f(a'))
\]

La proprietà di morfismo definita puntualmente per i morfismi verso gli esponenziali induce anche la struttura di poset su questi ultimi. Infatti se $\hat{f}\in\left[A,C^B\right]$, usando le strutture di poset su $A$, $C$ ed almeno formalmente quella ancora ignota su $C^B$, abbiamo che
\begin{align*}
(\forall b\in B){\mathcal G}(\hat{f}(-)(b))
& \quad\Leftrightarrow\quad
(\forall b\in B)(\forall a,a' \in A) (a \leq a' \Rightarrow \hat{f}(a)(b) \leq \hat{f}(a')(b)) \\
{\mathcal G}(\hat{f})
& \quad\Leftrightarrow\quad
(\forall a,a' \in A) (a \leq a' \Rightarrow \hat{f}(a) \leq \hat{f}(a'))
\end{align*}
e quindi ${\mathcal G}(\hat{f})\Leftrightarrow(\forall b\in B){\mathcal G}(\hat{f}(-)(b))$ implica che $\forall a,a' \in A$
\[
\hat{f}(a) \leq \hat{f}(a')
\quad\Leftrightarrow\quad
(\forall b\in B)(\hat{f}(a)(b) \leq \hat{f}(a')(b))
\]
proprio come ci aspetteremmo.

Per quanto detto in generale sappiamo già che una funzione il cui dominio è un prodotto è monotona se e solo se lo è ogni sua componente, ma qual è la struttura di poset che questa condizione induce sui prodotti? Essa è la congiunzione di
\[
(\forall a \in A)(\forall b,b' \in B)(b \leq b' \Rightarrow f(a,b) \leq f(a,b'))
\]
e
\[
(\forall b \in B)(\forall a,a' \in A)(a \leq a' \Rightarrow f(a,b) \leq f(a',b))
\]
ma ciò implica che scelti comunque $a,a' \in A$ e $b,b' \in B$ valga
\[
a \leq a' \wedge b \leq b'
\quad\Rightarrow\quad
f(a,b) \leq f(a',b')
\]
che, valutata sull'identità restituisce
\[
a \leq a' \wedge b \leq b'
\quad\Rightarrow\quad
(a,b) \leq (a',b')
\]
La monotonicità delle proiezioni produce l'implicazione inversa e così l'ordine sui prodotti è completamente fissato ed è quello che ci aspetteremmo.




\subsection*{Spazi metrici limitati e funzioni lipschitziane - REPRISE}

%Let's kick it. Non sto a ripeterlo in ogni proposizione ma le costanti di Lipschitz intendo che siano tutte numeri reali non negativi. 
La condizione di morfismo dunque sarà
\[
f \in [A,B]
\quad\Leftrightarrow\quad
{\mathcal G}(f)
\quad\Leftrightarrow\quad
(\exists K_f)(\forall a,a' \in A)(d_B(f(a),f(a')) \leq K d_A(a,a'))
\]

La condizione di {\it puntualizzabilità} \pol{La cosa avrà pure un nome vero, mi sa.} della proprietà di morfismo è che, per ogni $\hat{f}\in\left[A,C^B\right]$ valga
\[
(\forall b \in B)
(\exists K_{\hat{f},b})
(\forall a,a' \in A)
(d_C(\hat{f}(a)(b),\hat{f}(a')(b)) \leq K_{\hat{f},b} d_A(a,a'))
\]
se e soltanto se vale
\[
(\exists K_{\hat{f}})
(\forall a,a' \in A)
(d_{\left[B,C\right]}(\hat{f}(a),\hat{f}(a')) \leq K_{\hat{f}} d_A(a,a'))
\]

\hypertarget{p2c-lipshitz}{}
\pol[inline]{Probabilmente stando a districare i quantificatori uno trova la famiglia di metriche ammissibili. Siccome è una rogna, mi accontento di tagliare corto.}

\fou[inline]{S\`i, appunto; una possibile sbavatura di cui ti si potrebbe accusare \`e che questo metodo non definisce univocamente \emph{cosa} dovrebbe essere $d_{[B,C]}$: d'altra parte l'idea intuitiva \`e che la condizione metrica sia realizzata per $d_{[B,C]}(f,g):=\sup_{b}d_C(fb, gb)$; probabilmente uno determina anche questo smanettando coi quantificatori.  Resta inteso che uno dovrebbe verificare che $K_{\hat f} = \sup\dots$}
