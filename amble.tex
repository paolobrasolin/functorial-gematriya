\PassOptionsToPackage{dvipsnames}{xcolor}
\usepackage{amsmath,amssymb,
			wrapfig,amsthm, enumitem, 
			graphicx,url , stmaryrd, 
			todonotes,pifont,pigpen,marginnote}
\usepackage{mparhack}
\usepackage{makeidx}
\makeindex

\setlist{noitemsep}
\usepackage{tikz}

\usepackage[german,all,light,portrait]{draftcopy}
\def\K{\mathcal{K}}
\newcommand{\E}{\mathcal{E}}
\newcommand{\M}{\mathcal{M}}
\usepackage[left=4cm, right=4cm]{geometry}

\newcommand{\po}{\ar@{}[dr]|(.7){\text{\pigpenfont R}}}
\newcommand{\pb}{\ar@{}[dr]|(.3){\text{\pigpenfont J}}}
\newcommand{\pp}{\ar@{}[dr]|{\text{\pigpenfont N}}}
%\usepackage{newtxtext,newtxmath}
\usepackage{lmodern}
%\usepackage{bbold}
\usepackage[T1]{fontenc}
\usepackage[utf8]{inputenc}

\newcommand{\lort}{{}^\boxslash}
\newcommand{\rort}{^\boxslash}
%\renewcommand{\mathfrak}[1]{\text{\frakfamily #1}}

\newtheoremstyle{reference}%
   {}                %
   {}                %
   {}              %    Font del testo
   {}                      %    Rientro margini
   {\scshape}              %    Font del titolo dell'ambiente
   {.}                     %    Punteggiatura dopo "Teorema"\"Definizione"
   {.4em}                  %    Spazio tra titolo e testo.
   {\thmname{#1}           %    #1 : Definizione\Teorema\ecc
    \thmnumber{#2}         %    #2 : Contatore
    \thmnote{{\sc [#3]}}}  %    #3 : Testo tra "[" e "]"


\theoremstyle{definition}
	\newtheorem{df}{Definizione}[section]
	\newtheorem{thm}{Teorema}[section]
	\newtheorem{lem}{Lemma}[section]
	\newtheorem*{axio}{Assioma}
	\newtheorem{es}{Esempio}
	\newtheorem{prop}{Proposizione}[section]
	\newtheorem*{cor}{Corollario}
	\newtheorem{oss}{Osservazione}[section]
	\newtheorem{ex}{Esercizio}[section]
	\newtheorem{notat}{Notazione}[section]

%=====================================
%\let\amalg=\undefined
%\let\coprod=\undefined
%\DeclareSymbolFont{cmsymbols}{OMS}{cmsy}{m}{n}
%\DeclareSymbolFont{cmlargesymbols}{OMX}{cmex}{m}{n}
%\DeclareMathSymbol{\amalg}{\mathbin}{cmsymbols}{"71}
%\DeclareMathSymbol{\coprod}{\mathop}{cmsymbols}{"60}
%=================================================

\renewcommand{\mathbf}[1]{\text{\fontseries{b}\selectfont{#1}}}
\renewcommand{\textbf}[1]{\text{\fontseries{b}\selectfont{#1}}}
\def\C{\mathbf C}
\def\D{\mathbf D}

\newcommand{\coker}{\text{coker}}

\newcommand{\xto}[1]{\xrightarrow{#1}}
\newcommand{\xot}[1]{\xleftarrow{#1}}

\usepackage[all,cmtip]{xy}
\usepackage[polutonikogreek, italian]{babel}
\hyphenation{ag-giun-zio-ne o-mo-to-pi-che}
\usepackage{xcolor}

\DeclareMathAlphabet{\mathpzc}{OT1}{pzc}{m}{it}
	\newcommand{\com}[1]{\mathpzc{#1}}
\newcommand{\cate}[1]{\mathbf{#1}} 


\def\Sets{\cate{Sets}}
\def\Cat{\cate{Cat}}
\def\hocolim{\underrightarrow{{\rm holim}}}
\def\holim{\underleftarrow{{\rm holim}}}
\def\vcat{\com V\text{-}\cate{Cat}}

\usepackage{hyperref}
\hypersetup{backref,
	colorlinks=true,
	linkcolor=NavyBlue,citecolor=NavyBlue}
\usepackage{datetime}

\newcommand{\spc}{\textbf{Spc}}
\newcommand{\Top}{\textbf{Top}}
\usepackage[framemethod=TikZ]{mdframed}

\tikzstyle{titlered} =
    [draw=blue, thick, fill=blue,% 
        text=white, rectangle,  
        right, minimum height=.7cm]

\newcounter{exercise}

\usepackage{currfile}
\renewcommand*\theexercise{Exercise~\arabic{exercise}~(\currfilename)}

\makeatletter
\mdfdefinestyle{exercisestyle}{%
    outerlinewidth=1em,%
    outerlinecolor=white,%
    leftmargin=-1em,%
    rightmargin=-1em,%
    middlelinewidth=1.2pt,%
    roundcorner=5pt,%
    linecolor=blue,%
    backgroundcolor=blue!20,
    innertopmargin=1.2\baselineskip,
    skipabove={\dimexpr0.5\baselineskip+\topskip\relax},
    skipbelow={-1em},
    needspace=0.5\textheight,%3\baselineskip,
    frametitlefont=\sffamily\bfseries,
    settings={\global\stepcounter{exercise}},
    singleextra={%
        \node[titlered,xshift=1cm] at (P-|O) %
            {~\mdf@frametitlefont{\theexercise}~};},%
    firstextra={%
            \node[titlered,xshift=1cm] at (P-|O) %
                    {~\mdf@frametitlefont{\theexercise}~};},
}
\makeatother

\newenvironment{MyExercise}%
{\begin{mdframed}[style=exercisestyle,nobreak]}{\end{mdframed}}