Lo spazio etal\'e di un fascio.

L'equivalenza $\cate{Sh}(\underline X)\cong \Sets_{/X}$, e la presenza di un funtore $\cate{Top}_{/X}\to \cate{Sh}(X)$ suggerisce che si possa pensare un fascio su uno spazio $X$ come una sorta di ``spazio'' che lo ``riveste''.

Questo \`e vero. Si fa cos\`i.
\begin{MyExercise}
Mostrare che esiste una aggiunzione 
\[
\textsf{st}_x\dashv \textsf{sky}_x\colon \cate{PSh}(X) \leftrightarrows \Sets
\]
dove $\textsf{st}_x\colon \cate{PSh}(X)\to \Sets$ \`e il funtore che manda $F$ in
\[
\varinjlim_{U\supseteq \{x\}} F(U)
\]
e $\textsf{sky}_x$ \`e il funtore che manda un insieme $A$ nel prefascio
\[
\textsf{sky}_x(A) \colon U\mapsto
\begin{cases}
A & \text{ se } x\in U\\
* & \text{ altrimenti.}
\end{cases} 
\]
\end{MyExercise}
$\textsf{st}_x(F)$ \`e la \emph{spiga} del prefascio $F$.
\begin{MyExercise}
Mostrare che $\textsf{sky}_x(A)$ \`e un fascio, sicch\'e l'aggiunzione si restringe a $\cate{Sh}(X)$; $\textsf{st}_x(F)=F_x$ \`e la spiga del fascio $F$. 
\end{MyExercise}
Le propriet\`a di un fascio sono altamente determinate dalle propriet\`a congiunte dell'insieme delle sue spighe:
\begin{MyExercise}
Mostrare che un morfismo $\eta\colon F\Rightarrow G$ tra fasci su $X$ \`e un iso/mono/epi se e solo se lo \`e ciascun $\textsf{st}_x\eta\colon F_x\to G_x$.
\end{MyExercise}
