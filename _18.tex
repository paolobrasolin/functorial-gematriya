\begin{MyExercise}
Sia $X$ uno spazio topologico. Mostrare che la corrispondenza che associa $U\subseteq X$ aperto all'insieme
\[
\mathcal C^0(U) = \{f\colon U\to \mathbb{R}\mid \text{ continua}\}
\]
\`e un funtore $\mathcal C^0\colon \cate{Op}(X)\to \Sets$, che soddisfa le seguenti due propriet\`a:
\begin{itemize}
\item Per ogni $U\in\cate{Op}(X)$ e ogni ricoprimento aperto $\{U_i\}$ di $U$, chiamando $\rho=\rho_{VW}=F(W\subseteq V)$ la funzione indotta tra $F(V)$ ed $F(W)$, si ha
\[
\forall i\in I\big(\rho_{U, U_i}(f) = \rho_{U, U_i}(g)\big)\quad \Rightarrow \quad f=g
\]
\item Per ogni $U\in\cate{Op}(X)$ e ogni ricoprimento aperto $\{U_i\}$ di $U$, data una famiglia di funzioni continue $\{f_i\colon U_i\to \mathbb R\}$ tali che
\[
\rho_{U_i, U_{ij}}(f_i) = \rho_{U_j, U_{ij}}(f_j)
\]
per ogni $i,j\in I$, allora esiste una funzione continua $f\colon U\to \mathbb R$ (e come corollario del primo punto, \`e unica) tale che $\rho_{U, U_i}(f)= f_i$ per ogni $i\in I$.
\end{itemize}\end{MyExercise}
Queste condizioni sono le \emph{condizioni di fascio}.
\begin{df}
Ogni funtore controvariante $F\colon \cate{Op}(X)\to \Sets$ che soddisfi le stesse due propriet\`a, per ogni elemento dell'insieme $F(U)$, si dice un \emph{fascio} su $X$.
\end{df}

Denotiamo con $i_{U,V}$ l'unico morfismo di $\cate{Op}(X)(U,V)$, ovvero la mappa (continua) di inclusione. Allora $\rho_{V,U}=C^0(i_{U,V})\colon C^0(V)\to C^0(U)$ è la precomposizione con $i_{U,V}$ e manda una funzione continua di dominio $V$ nella sua restrizione ad $U$, che denoteremo $f\vert_U=f\circ i_{U,V}$.

Scriviamo in modo un po' più suggestivo le proprietà da verificare.

La prima. Dato comunque un aperto $U\in\cate{Op}(X)$ ed un suo ricoprimento aperto $\{U_i\}_{i\in I}$ e due funzioni continue $f,g\in C^0(U)$,
\[
(\forall i\in I) \left(f\vert_{U_i}=g\vert_{U_i}\right) \Rightarrow (f=g)
\]
ovvero, se due funzioni continue coincidono ristrette ad ognuno degli aperti di un ricoprimento del dominio allora coincidono sull'intero dominio.
È immediata da dimostrare, negata e srotolata:
\[
(\exists x\in U)(f(x)\neq g(x)) \Rightarrow (\exists i\in I)(\exists y\in U_i)(f\vert_{U_i}(y)\neq g\vert_{U_i}(y))
\]
infatti basta scegliere $i$ tale che $x\in U_i$, ovviamente possibile per le proprietà del ricoprimento, ed $y=x$.

La seconda.
Dato comunque un aperto $U\in\cate{Op}(X)$ ed un suo ricoprimento aperto $\{U_i\}_{i\in I}$ ed una famiglia di funzioni continue $\{f_i\colon U_i\to \mathbb R\}$,
\[
(\forall i,j\in I) \left(f_i\vert_{U_i\cap U_j}=f_j\vert_{U_i\cap U_j}\right) \Rightarrow (\exists! f\in C^0(U)\colon (\forall i\in I)(f\vert_{U_i}=f_i))
\]
La dimostrazione è leggermente più delicata.
L'unicità di $f$ è immediata grazie alla prima proprietà.
L'esistenza si potrebbe avere definendola come $f(x)=f_i(x)$ per ogni $x\in U_i$, possibile proprio per la condizione di compatibilità sulle intersezioni.
Ma la continuità?
Vale la pena di dedicare un poco di spazio al

{\lem[Pasting] Siano $X$ ed $Y$ due spazi topologici, $\{U_i\}_{i\in I}$ un ricoprimento aperto di un aperto $U\subseteq X$ ed $\{f_i\colon U_i\to Y\}_{i\in I}$ una famiglia di funzioni continue.
Se $f_i\vert_{U_i\cap U_j}=f_j\vert_{U_i\cap U_j}$ per ogni $i,j\in I$ allora esiste un'unica mappa continua $f\colon U\to Y$ tale che $f\vert_{U_i}=f_i$ per ogni $i\in I$.}
{\proof Mostriamo che la funzione esiste, è unica ed ha la proprietà cercata.
Imponiamo quest'ultima puntualmente definendo $f(x)=f_i(x)$ con un $i$ tale che $x\in U_i$.
Un tale $i$ esiste grazie alle proprietà del ricoprimento, e se la scelta non è univoca è comunque irrilevante grazie alla compatibilità sulle intersezioni chiesta per ipotesi.
Inoltre questa è l'unica definizione possibile che rispetti la proprietà di restrizione.

Occupiamoci ora della continuità.
Se $V$ è un aperto di $Y$ allora per definizione di continuità $f^{-1}_i(V)$ è un aperto di $U_i$.
La preimmagine è anche un aperto di $U$, poiché è subset aperto di un subset aperto di $U$.
Per la definizione di $f$ possiamo scrivere l'uguaglianza
\[
f^{-1}(V)=\bigcup_{i\in I}f^{-1}_i(V)
\]
ma siccome un'unione arbitraria di aperti è aperto concludiamo. \qed}
\fou[inline]{S\`i; probabilmente $Y=\mathbb R$ per te, ma ovviamente i numeri reali non hanno un ruolo davvero particolare. }

La proprietà cercata segue immediatamente.

\pol[inline]{In effetti dalla seconda proprietà segue la prima.
Basta pensare alle funzioni come definite univocamente dal pasting delle restrizioni agli aperti di un ricoprimento.
Dunque nel caso del fascio delle funzioni continue tra aperti di uno spazio topologico la seconda proprietà garantisce la prima.
Curioso!
Naturalmente questo sarà falso in generale.}

\pol[inline]{No, aspetta.
Supponiamo che la seconda condizione di fascio valga. Allora la famiglia $\{f\vert_{U_i}\colon U_i\to \mathbb R\}_{i\in I}$ delle restrizioni di $f\colon U\to \mathbb R$ agli aperti del ricoprimento $\{U_i\}_{i\in I}$ determina $f$ univocamente. Dunque se un'altra funzione $g$ ha la medesima restrizione, aperto per aperto, ho $g=f$. Ovvero, la prima condizione.
Mi sta sfuggendo qualcosa?}

\pol[inline]{Risolto. La prima implica l'unicità della seconda. E infatti è più corretto scrivere la seconda senza unicità! (Warning: la consegna è stata corretta.)}
