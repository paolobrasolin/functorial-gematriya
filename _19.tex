\begin{MyExercise}
Mostrare che le seguenti corrispondenze sono fasci su uno spazio topologico $X\supseteq U$:
\begin{itemize}
\item $U\mapsto \mathcal C^k(U)$ per $k\in [0,\infty]\cup\{\omega\}$ se $X$ \`e una variet\`a differenziabile di classe $\mathcal C^k$;
\item $U\mapsto H(U) =$ funzioni olomorfe su $U$, se $X$ \`e una variet\`a complessa;
\item $U\mapsto \Omega^p(U) = \{p\text{-forme differenziali su } U\}$ se $X$ \`e una variet\`a differenziabile di dimensione finita, e $p\in\{0,\dots, \dim X\}$;
\item $U\mapsto \mathfrak{Der}(U)$, le \emph{derivazioni} su $U$, ovvero le mappe $\mathcal C^\infty(U)\to \mathcal C^\infty(U)$ che sono lineari e Leibniz.
\end{itemize}
Mostrare che le seguenti corrispondenze non sono fasci su $X$:
\begin{itemize}
\item $U\mapsto \{f\colon U\to\mathbb R\mid |f(x)| < M_f\; \forall x\in U\}$ se $X=\mathbb R^n$, $n\ge 1$;
\item $U\mapsto \faktor{H(U)}{\text{im }\frac{\partial}{\partial z}}$ se $X = \mathbb C\setminus \{0\}$.
\end{itemize}
\end{MyExercise}




\subsubsection*{Funzioni differenziabili su varietà differenziabili}

Una varietà si presenta come uno spazio topologico secondo numerabile ed Hausdorff munito di un atlante.
Un atlante è un ricoprimento i cui aperti siano muniti ognuno di un omeomorfismo (carta) verso $\mathbb R^n$, per qualche $n$ che si dice dimensione della varietà.

Gli atlanti catturano l'idea di essere {\it localmente omeomorfi ad $\mathbb R^n$}.
L'hausdorfficità non è misteriosa.
Secondo numerabile vuol dire che la topologia deve avere una base numerabile.
È una richiesta molto forte, ma non ne ho mai colto la necessità.
\pol[inline]{Vorrò una giustificazione fascio-teoretica delle proprietà topologiche richieste, se esiste.}

Questa è una definizione piuttosto grezza di varietà topologica. Arricchendola di una nozione di differenziabilità si ottengono le varietà differenziabili. Porchettiamo per capire come dovrebbero essere fatte usando l'unica cosa che abbiamo: la differenziabilità negli spazi euclidei. La ragionevolezza suggerisce un'unica definizione.

Siano $X$ ed $Y$ due varietà topologiche di dimensioni $m$ ed $n$.
Siano $U\subseteq X$ e $V\subseteq Y$ due loro aperti.
Diremo che una funzione $f\colon U\to V$ è di classe $\mathcal C^k$ in $p\in U$, i.e.
\[ f\in\mathcal C^k_p(U,V) \]
se scelte comunque due carte $\alpha\colon A\to\mathbb R^m$ in $X$ e $\beta\colon B\to\mathbb R^n$ in $Y$, tali che $p\in A$ ed $f(p)\in B$, vale
\[ \beta\circ f\circ\alpha^{-1}\in\mathcal C^k_{\alpha(p)}(\alpha(A\cap U),\beta(B\cap V)) \]
Diremo che $f$ è di classe $\mathcal C^k$ se lo è in ogni punto.
Ovvero $f\in\mathcal C^k(U,V)$ se $\beta\circ f\circ\alpha^{-1}\in\mathcal C^k(\alpha(A\cap U),\beta(B\cap V))$ per ogni coppia di carte.

Non avevo mai visto definire le cose in questo ordine prima.
Trovo che chiarifichi l'usuale definizione di varietà differenziabile di classe $\mathcal C^k$.
Infatti essa ammonta a chiedere che l'identità sia $\mathcal C^k$.
Ancora meno misterioso della richiesta di differenziabilità sulle mappe di transizione, a mio parere.
\pol[inline]{Perché non me l'aveva detto nessuno? In questo modo la definizione actually makes sense!}

Dalla definizioni date è immediato che la composizione di due funzioni, una $\mathcal C^p$ ed una $\mathcal C^q$, è almeno di classe $\mathcal C^{\min(p,q)}$.

Da ora lavoriamo in un un'unica varietà differenziabile $X$ di classe $\mathcal C^k$.
Denotiamo con $i_{U,V}$ l'unico morfismo di $\cate{Op}(X)(U,V)$, ovvero la mappa di inclusione.
Allora $\mathcal C^k(i_{U,V})\colon \mathcal C^k(V)\to \mathcal C^k(U)$ è la precomposizione con essa.

In una varietà $\mathcal C^k$ le inclusioni sono $\mathcal C^k$ (si tratta di restrizioni dell'unità, ovvero delle mappe di transizione). La precomposizione preserva la classe di differenziabilità e quindi anche l'operazione di restrizione la preserva: $\mathcal C^k$ allora è proprio un funtore ed utilizzeremo l'usuale notazione per le restrizioni.

La condizione di fascio è la seguente. Per ogni aperto $U\subseteq X$ ed ogni famiglia di funzioni $f_i\in\mathcal C^k(U_i)$ con $i\in I$ tale che i domini siano un ricoprimento aperto di $U$,
\[
(\forall i,j\in I) \left(f_i\vert_{U_i\cap U_j}=f_j\vert_{U_i\cap U_j}\right) \Rightarrow (\exists! f\in C^k(U)\colon (\forall i\in I)(f\vert_{U_i}=f_i))
\]
L'esistenza è immediata definendo $f$ puntualmente identica alle $f_i$, evitando ambiguità grazie alla condizione di compatibilità.
L'unicità deriva dal fatto che è l'unica definizione possibile che realizza la condizione richiesta ad $f$.
Inoltre, la classe di differenziabilità è una proprietà che abbiamo definito puntualmente quindi $f$ la eredita dalle $f_i$ per costruzione.

\pol[inline]{In effetti bastava usare la definizione locale di continuità nell'esercizio precedente per abbreviarlo.}




\subsubsection*{Funzioni olomorfe su varietà complesse}

Procediamo per analogia col caso precedente.
Costruiamo un prototipo di varietà complessa di dimensione $n$ come uno spazio topologico secondo numerabile ed Hausdorff, localmente omeomorfo al cerchio unitario aperto di $\mathbb D^n\subset\mathbb C^n$.

È naturale tentare di definire la nozione di funzione olomorfa tra oggetti di questo tipo.
Prendiamo $X$ ed $Y$ di dimensioni $m$ ed $n$ e due aperti $U\subseteq X$ e $V\subseteq Y$.
Diremo che una funzione $f\colon U\to V$ è olomorfa in $p\in U$, i.e.
\[ f\in H_p(U,V) \]
se scelte comunque due carte $\alpha\colon A\to\mathbb D^m$ in $X$ e $\beta\colon B\to\mathbb D^n$ in $Y$, tali che $p\in A$ ed $f(p)\in B$, vale
\[ \beta\circ f\circ\alpha^{-1}\in H_{\alpha(p)}(\alpha(A\cap U),\beta(B\cap V)) \]
Diremo che $f$ è olomorfa se lo è in ogni punto.
Ovvero $f\in H(U,V)$ se $\beta\circ f\circ\alpha^{-1}\in H(\alpha(A\cap U),\beta(B\cap V))$ per ogni coppia di carte.

Ricordiamo che una funzione $h\colon Y\subseteq\mathbb C^m\to Z\subseteq\mathbb C^n$ è olomorfa in un punto se in esso si annulla la derivata di Wirtinger di ogni sua componente. Sarà olomorfa se tale condizione vale sull'intero dominio.

Definiamo dunque una varietà complessa come un oggetto di questo genere tale che l'identità sia olomorfa.
\pol[inline]{No, davvero. Qui sta succedendo {\it qualcosa}.}

Riguardo alle condizioni di fascio?
La regola della catena garantisce che la composizione preservi l'olomorficità.
Il resto delle verifiche sono una riscrittura dell'esercizio precedente poiché abbiamo definito tutte le proprietà localmente.

\pol[inline]{Holy sh--- io induco una proprietà definita localmente ed allora se ce l'ha l'identità ce l'hanno tutte le mappe tra aperti!}

\hrulefill

Un \emph{incubo} su $X$ consta di un oggetto di $\Top_{/X}$; i morfismi $(E, h)\to (E', h')$ di $\Top_{/X}$ sono dati da triangoli commutativi
\[
\xymatrix{
E \ar[dr]_h\ar[rr]^f && E'\ar[dl]^{h'}\\
& X
}
\]
Una \emph{sezione} di un incubo $(E,h\colon E\to X)$ su $X$ consiste di un elemento generalizzato di $(E, h)$, ovvero di un morfismo $(X, 1_X)\to (E, h)$.
\begin{MyExercise}
Mostrare che per ogni inclusione di un aperto $i\colon U\subseteq X$, la corrispondenza
\[
U\mapsto 
\left\{
\left. 
\begin{tikzcd}
U \arrow[d, "s"] \\
E
\end{tikzcd}
\right|
\begin{tikzcd}
U \arrow[rr, "s"] \arrow[dr, "i", hook]&& E \arrow[dl, "h"] \\
& X &
\end{tikzcd}
\right\}
\]
dove $i\colon U\subseteq X$, definisce un fascio su $X$.
\end{MyExercise}

Andiamo in $\cate{Top}$. Prendiamo un fibrato $p:E\to B$.
Possiamo restringerlo ad un aperto $U\subseteq B$ con un pullback lungo l'inclusione:
\[\xymatrix{
p^\leftarrow(U) \ar[d]_{i^\star_{UB}p} \ar[r] & E \ar[d]^p \\
U \ar[r]_{i_{UB}} & B
}\]

Consideriamo il funtore $S_p:\cate{Op}(B)\to \Sets$ che manda $U$ nella collezione delle sezioni di $i^\star_{UB}p$. Esso manda un'inclusione $i$ nella funzione agente come la precomposizione $s\mapsto s\circ i$ sulle sezioni. Le verifiche di funtorialità sono immediate.

$S_p$ è il fascio delle sezioni locali del fibrato $p$.
Ma è davvero un fascio?
Vogliamo che per ogni aperto $U$ di $B$ ed ogni famiglia di sezioni $s_i\in S_p(U_i)$ indiciata da $i\in I$ i cui domini formino un ricoprimento aperto di $U$ valga
\[
(\forall i,j\in I)(f_i\circ i_{U_{ij},U_i}=f_j\circ i_{U_{ij},U_j})
\Rightarrow
(\exists! f\in S_p(U))(\forall i\in I)(f\circ i_{U,U_i}=f_i)
\]
dove denotiamo $U_{ij}=U_i\cap U_j$.
Esistenza, unicità e continuità di $f$ si mostrano esattamente come già fatto in precedenza.
Basta solo verificare che $f\in S_p(U)$, ovvero $i^\star_{UB}p\circ f=\mathrm{id}_U$, ma ciò è vero puntualmente per definizione.



\subsubsection*{Forme differenziali su varietà differenziabili}

Data una varietà $X$ consideriamo il fibrato $\Lambda^p(T^\star X)\to X$ (cioè dalla $p$-esima potenza esterna del fibrato cotangente).
Il funtore $\Omega^p$ è esattamente il {\it fascio} delle sue sezioni locali.

\pol[inline]{Was that too cheap?}




\subsubsection*{Derivazioni}

\pol[inline]{Mi sto incaloppiando su come restringere le derivazioni - non sul senso, ma su come scriverlo bene. Ci devo pensare un attimo.}




\subsubsection*{Prefascio delle funzioni localmente limitate}

%\pol[inline]{Credo ci sia una svista nella consegna. Un $X$ al posto di un $U$.}

Consideriamo le funzioni $f_i=\pi_1\vert_{B_i}$ dove intendiamo che $B_i$ sia la palla aperta di centro $(i,0,\ldots)$ e raggio unitario, per $i\in\mathbb N$.
Esse sono tutte limitate ed il loro incollamento esiste unico ma non è limitato, quindi non abbiamo un fascio.

Moralmente quello che succede è che sto rendendo troppo rigidi i pezzi per gli incollamenti. Difatti l'ostruzione è che l'incollamento non esiste. Cioè, viene meno la seconda condizione di fascio.


\subsubsection*{Prefascio delle funzioni olomorfe sminquiate}

Consideriamo per $X = \mathbb C\setminus \{0\}$ il seguente prefascio:
\[U\overset{F}\mapsto \faktor{H(U)}{\textstyle \text{im }\frac{\partial}{\partial z}}\]

L'olomorficità equivale a richiedere l'esistenza di antiderivate locali.
Il set a denominatore del quoziente contiene esattamete le funzioni che hanno antiderivata globale.
Quindi, questo prefascio assegna ad $U$ le funzioni complesse che hanno antiderivata locale ma non globale.

Moralmente quello che succede è che sto rendendo troppo flessibili i pezzi per gli incollamenti. Infatti mi aspetto che venga meno la prima condizione di fascio, cioè l'unicità dell'incollamento.

Il modo più rapido è mostrare due funzioni differenti con restrizioni agli aperti di un ricoprimento del dominio identiche.

\pol[inline]{Hnnng. Non mi viene, ora.}

\fou[inline]{La topologia (più precisamente il gruppo fondamentale, o il primo gruppo di (co)omologia) di $U$ dice quali forme differenziali hanno soluzione o no, sia in ambito reale che in ambito complesso; in questo caso è utile osservare che la condizione di separazione si enuncia così:
\begin{quote}
Per ogni $U\in\cate{Op}(X)$, e ogni ricoprimento aperto $\{U_i\}$ di $U$, se $s\in F(U)$ è zero su ogni $U_i$, $s|_{U_i}=0$, allora $s=0$
\end{quote}
(è immediato: $s|_{U_i}=t|_{U_i} \iff (s-t)|_{U_i}=0$). Questo ti sta dicendo che la domanda si rifrasa come: per il prefascio dato, è vero che per ogni $U$, e ogni ricoprimento di tale $U$, quando $s|_{U_i}=0$ allora $s=0$? La risposta è no: prendi $U=\mathring{D}(0,1[$ (il disco bucato di centro l'origine) e come $s\in FU$ prendi la funzione $\frac{1}{z}$; allora, su ogni ricoprimento di tale $U$ la funzione è la derivata di qualcuno, e quindi diventa zero in quel conucleo: scegli un ramo di $\log z$. E però --dimostralo o ricordatelo da Metodi!-- globalmente $s\neq 0$, ovvero $\frac{1}{z}$ non è la derivata di nessuno.}