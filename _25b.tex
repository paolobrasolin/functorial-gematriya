\begin{df}
Un incubo $p\colon E\to X$ \`e un \emph{omeomorfismo locale} se soddisfa la seguente propriet\`a:
\begin{quote}
per ogni $y\in E$ esiste un intorno aperto $U_y\subseteq E$ tale che $p|_{U_y}$ \`e un omeomorfismo sull'immagine: $p|_{U_y}\colon U_y\cong p(U_y)$.
\end{quote}
\end{df}
\begin{MyExercise}
\begin{itemize}
\item Se $p\colon E\to X$ \`e un omeomorfismo locale, $p$ \`e una mappa aperta (manda aperti in aperti);
\item Un incubo $p\colon E\to X$ su $X$ \`e un omeomorfismo locale se e solo se $p$ e $\langle p,p\rangle\colon E\times_X E\to X$ sono mappe aperte.
\item Se $U\in \cate{Op}(X)$, l'insieme
\[
\{ s(U)\mid s\in\Gamma_p (U),\; U\in\cate{Op}(X)\}
\]
\`e una base di aperti per la topologia di $E$;
\item Se $\varphi\colon (E,p)\to (F, q)$ \`e un morfismo di incubi che sono omeomorfismi locali, anche $\varphi$ \`e un omeomorfismo locale.
\end{itemize}
\end{MyExercise}
