\subsubsection*{Slice ed aggiunzioni}

Sia $\C$ una categoria con tutti i pullback.
Una sua freccia $f\colon X\to Y$ induce un funtore pullback $f^\star\colon \C/Y\to\C/X$. Diciamo che sugli oggetti agisce come il pullback lungo $f$ e consideriamo il diagramma seguente per capire come agisce su una freccia $\C/Y$:
\[\xymatrix{
= \ar[d] \ar@{.>}[r] \ar@(ur,ul)[rr]_{f^\star g}
	& - \ar[d] \ar[r]_{f^\star h}
	& X \ar[d]^f \\
A \ar[r]^m \ar@(dr,dl)[rr]^{g}
	& B \ar[r]^h
	& Y
}\]
Il quadrato destro e quello bombato sono pullback. Allora per la commutatività del triangolo inferiore e la proprietà {\it due su tre} sappiamo che il quadrato sinistro è un pullback ed il triangolo superiore commuta. In simboli:
\[ f^\star g = f^\star (hm) = f^\star h \circ (h^\star f)^\star m \]
Ora sappiamo anche come deve agire il funtore sui morfismi:
\[ f^\star(m\colon g\to h)=((h^\star f)^\star m\colon f^\star g\to f^\star h) \]

Consideriamo ora un funtore $L\colon \C\to\D$ ed un'aggiunzione $L\dashv R$. Possiamo provare a definire un funtore $L/C\colon \C/C\to\D/LC$ nel modo ovvio su oggetti e morfismi:
\[
\xymatrix{
X \ar[d]_x \ar[r]^f & Y \ar[dl]^y \\ C
}
\qquad\mapsto\qquad
\xymatrix{
LX \ar[d]_{Lx} \ar[r]^{Lf} & LY \ar[dl]^{Ly} \\ LC
}
\]
Usare ingenuamente $R$ come aggiunto destro ci fa finire in $\C/RLC$. Ma allora basta usare l'unità dell'aggiunzione $\eta\colon RL\to1_\C$ per definire il funtore pullback lungo la sua componente $C$, ovvero $\eta_C^\star\colon \C/RLC\to\C/C$. Postcomponendolo ad $R$ individuiamo univocamente un triangolo che è un morfismo in $\C/C$:
\[
\xymatrix{
W \ar[d]_w \ar[r]^g & Z \ar[dl]^z \\ LC
}
\qquad\mapsto\qquad
\xymatrix{
RW \ar[d]_{Rw} \ar[r]^{Rg} & RZ \ar[dl]^{Rz} \\ RLC
}
\qquad\mapsto\qquad
\xymatrix{
= \ar[d]_{\eta_C^\star Rw} \ar@{.>}[r] & - \ar[dl]^{\eta_C^\star Rz} \\ C
}
\]
\pol[inline]{Resta da vedere che è davvero un'aggiunzione. Mah. Dev'esserci un modo meno grezzo.}
\pol[inline]{Meh. Faccio il conto e buonanotte.}

Fissiamo un set $X$.

L'oggetto terminale di $\Sets_{/X}$ è $\mathrm{id}_X$ poiché è l'unico che fa commutare qualsiasi triangolo della condizione di morfismo.

La definizione di prodotto in $\Sets_{/X}$ coincide con quella di pullback in $\Sets$. Vale a dire che che il prodotto di $f\colon A\to X$ e $g\colon B\to X$ è $f\otimes g\colon A\times_X B\to X$, definito dalle composizioni con le proiezioni: $f\otimes g=fp_A=gp_B$.

Se $\Sets_{/X}$ è cartesiana allora mi aspetto che $\mathrm{id}_X$ sia l'identità ed il prodotto sia dato dalla diagonale del pullback come appena spiegato. Ed in effetti è immediato che il pullback di un morfismo lungo un'identità dia il morfismo stesso: $\mathrm{id}_X\times f=f$.

\pol[inline]{Adesso a me basterebbe avere l'aggiunto destro del {\it funtore pullback}, che infatti era la costruzione che avevo tentato di fare inizialmente.}




\subsubsection*{Esperimento-time!}

Vado in $\cate{Top}$.
Qui posso fare restrizioni come pullback lungo le inclusioni.
Denotiamole come $i_{UV}\colon U\to V$.
Sia invece $f^\star$ il funtore pullback lungo la freccia $f$.

Un fibrato su $B$ è un epi $p\colon E\to B$ tale che per ogni punto $x\in B$ esistono un suo intorno $U$ (aperto banalizzante), uno spazio $F_x$ (fibra) ed un omeomorfismo $h_x$ (trivializzazione locale) tali da far commutare
\[\xymatrix{
p^\leftarrow(U) \ar[d]_{i_{UB}^\star p} \ar[r]^{h_x} & U\times F_x \ar[dl]^{\pi_U} \\
U
}\]

Un aperto è intorno di ogni suo punto quindi dal diagramma segue che tutti gli spazi $F_x$ relativi ai punti di un medesimo aperto banalizzante sono omeomorfi.
Smettiamo di distinguerli - ovvero, li identifichiamo con la loro classe di omeomorfismo.

Consideriamo due fibrati $p\colon E\to B$ e $q\colon F\to C$ e rispettive trivializzazioni locali. In diagrammi:
\[\xymatrix{
U\times F_U \ar[dr]_{\pi_U} & p^\leftarrow(U) \ar[l]_h \ar[d]^{i_{UB}^\star p} &
q^\leftarrow(V) \ar[d]_{i_{VC}^\star q} \ar[r]^k & V\times G_V \ar[dl]^{\pi_V} \\
& U & V
}\]

Siano $p=q$, $h=h_i$, $k=h_j$, $U=U_j$, $V=U_i$ ed $U_{ij}=U_i\cap U_j$. Possiamo incollare i pullback dei due triangoli lungo le inclusioni in $U_{ij}$. Siccome le fibre sono omeomorfe nell'intersezione, otteniamo (lascio implicite le ovvie restrizioni)
\[\xymatrix{
U_{ij}\times F_{U_{ij}} \ar[r]^{h_j^{-1}} \ar[dr]_{\pi_{U_{ij}}} & p^\leftarrow(U_{ij}) \ar[d]|{i_{U_{ij}B}^\star p} \ar[r]^{h_i} & {U_{ij}}\times F_{U_{ij}} \ar[dl]^{\pi_{U_{ij}}} \\
& U 
}\]
Ciò significa che
\[
h_i\circ h_j^{-1}\colon(x,v)\mapsto(x,g_{ij}(x)(v))
\]
dove $g_{ij}$ è una mappa continua che manda punti di $U_{ij}$ in mappe continue su $F_{U_{ij}}$.

Morale: punti dello stesso aperto banalizzante hanno fibre omeomorfe; trivializzazioni locali distinte associano fibre omeomorfe al medesimo punto. Everything works, insomma.

Ora. Sia $B$ uno spazio topologico secondo numerabile ed Hausdorff.

\pol[inline]{uh, sì. praticamente posso ricostruire la definizione di varietà. eccetera}


Allarghiamo le maglie. Consideriamo il seguente diagramma:
\[\xymatrix{
U\times F_U \ar[dr]_{\pi_U} & p^\leftarrow(U) \ar[l]_h \ar[d]^{i_{UB}^\star p} \ar[r]^{\hat{f}} &
q^\leftarrow(V) \ar[d]_{i_{VC}^\star q} \ar[r]^k & V\times G_V \ar[dl]^{\pi_V} \\
& U \ar[r]_f & V
}\]
Fissato $\hat f$, ogni sezione di $\pi_U$ determina un unico $f$ che renda il diagramma commutativo

Per definizione di fibrato gli aperti banalizzanti sono anche un ricoprimento aperto dello spazio di base.
D'altra parte essi sono compatibili sulle intersezioni nel senso visto.
Fa venire voglia di incollarli.
